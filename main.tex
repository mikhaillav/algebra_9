\documentclass[12pt, a4paper]{report}

% Импорты 

\usepackage{cmap}
\usepackage{url}
\usepackage{hyperref}
\usepackage{amsthm, amsmath, amssymb}
\usepackage{thmtools}
\usepackage{fancyhdr}
\usepackage{graphicx}
\usepackage{titlesec}
\usepackage[normalem]{ulem}
\usepackage{cancel}
\usepackage{subcaption}

\usepackage[english, russian]{babel}
\usepackage[utf8]{inputenc}
\usepackage[T2A]{fontenc}

% Геометрия 

\usepackage{geometry}
\geometry{
	left=1.5cm,
	right=1.5cm,
	top=2cm,
	bottom=1.5cm
}

\setlength\parindent{0pt}

% Теоремы, леммы ...

\renewcommand{\listtheoremname}{Список теорем}

\newtheorem{theorem}{Теорема}[section]
\newtheorem{lemma}[theorem]{Лемма}
\newtheorem{corollary}[theorem]{Следствие}
\theoremstyle{definition}
\newtheorem*{definition}{Определение}
\theoremstyle{remark}
\newtheorem{remark}[theorem]{Замечание}
\newtheorem{example}[theorem]{Пример}
\theoremstyle{remind}
\newtheorem*{remind}{Remind}

\newcommand{\divisible}{\mathop{\raisebox{-2pt}{\smash{\vdots}}}}
\newcommand{\notdivisible}{\mathop{\raisebox{-2pt}{$\not \vdots$ }}}

% Колонтитулы

\pagestyle{fancy}
\fancyhf{}
\fancyhead[R]{\rightmark}
\fancyhead[L]{\textsc{\thechapter. \leftmark}}
\fancyfoot[C]{\thepage}
\renewcommand{\chaptermark}[1]{\markboth{#1}{}}
\renewcommand{\sectionmark}[1]{\markright{#1}}

% Главы

\titleformat{\chapter}[display]
{\normalfont\huge\bfseries\centering} % Формат текста
{\titlerule[1pt] \vspace{1pt} \titlerule\vspace{1pc} \Huge\MakeUppercase{\chaptertitlename} \thechapter} % Номер главы
{1pc} % Отступ между номером главы и названием
{\Huge} % Формат названия главы
[\vspace{1pc} \titlerule] % Декоративные элементы после названия

\titlespacing*{\chapter}{0pt}{0pt}{40pt}

\begin{document}
	
	\begin{titlepage}
		\newgeometry{left=1.5cm, right=1.5cm, top=3cm, bottom=2cm}
		\centering
		
		{\Large Санкт-Петербургский губернаторский\\ Физико-математический лицей №30}\\[1cm]
		\includegraphics[width=0.18\textwidth]{assets/logo.png}\\[5cm]
		
		\textbf{ \LARGE Конспект уроков алгебры 9-го класса}\\[2cm]
		
		\begin{flushright}
			\textbf{Ученика:} \\ Лавелина М.Е. \\[1cm]
			\textbf{По мотивам уроков:} \\ Ренёва О.В. \\[1cm]
		\end{flushright}
		
		\vfill
		
		{\large 2024--2025 г.}
		
	\end{titlepage}
	
	\begin{abstract}
		Я пишу для себя так, понятно и удобно мне. Некоторые формулировки могут быть не точными или вовсе не верными. 
		Думайте прежде всего \textbf{своей} головой. \\

		
		\begin{center}
		  Удачи в изучении алгебры!
		\end{center} 
		
	  \end{abstract}

	\include{tex/contents.tex}	
	\chapter{Степени и корни}

\section{Степень с целым показателем}

\begin{definition}
    \[
        m \in \mathbb{Z}: \; a^m = 
        \left\{
        \begin{array}{l}
            a^m, \; m \in \mathbb{N} \\ \\
            1, \; m = 0 \\ \\
            \frac{1}{a^m}, \; m \in  \mathbb{Z} \land n < 0 \\
        \end{array}\right.
    \]
\end{definition}

\begin{remark}
    $a^m$, при $m \in \mathbb{Z} \; \backslash \; \mathbb{N}$ не определенно при $a = 0$. Иными словами, ноль нельзя возводить в неположительную степень.
\end{remark}

\begin{example}
    $2^{-3} = \frac{1}{2^3} = \frac{1}{8}$; $(\frac{3}{2})^{-2} = \frac{2^2}{3^2} = \frac{4}{9}$;
\end{example}

\begin{theorem}
    $\forall a \in \mathbb{R} \backslash \{0\} \; \forall m, n \in \mathbb{Z}: \; a^m \cdot a^n = a^{m + n}$
\end{theorem}

\begin{proof}
    \hfill

    \begin{enumerate}
        \item $m,n \in \mathbb{N}$ - см. доказательство для $\mathbb{N}$
        \item $m = 0 \lor n = 0$ - очевидно
        \item $m < 0 \land n > 0$:
        \begin{align*}
            a^m \cdot a^n = \frac{a^n}{a^{-m}} = a^{n - (-m)} = a^{m + n}
        \end{align*}
        \item $m > 0 \land n < 0$:
        \begin{align*}
            a^m \cdot a^n = \frac{a^m}{a^{-n}} = a^{m - (-n)} = a^{m + n}
        \end{align*}
        \item $m < 0 \land n < 0$:
        \begin{align*}
            a^m \cdot a^n = \frac{1}{a^{-n} \cdot a^{-m}} = \frac{1}{a^{-(n+m)}} = a^{m + n}
        \end{align*}
    \end{enumerate}
\end{proof}

\break

\begin{theorem}
    $\forall a \in \mathbb{R} \backslash \{0\} \; \forall m, n \in \mathbb{Z}: \; (a^m)^n = a^{m \cdot n}$
\end{theorem}

\begin{proof}
    \hfill

    \begin{enumerate}
        \item $m,n \in \mathbb{N}$ - см. доказательство для $\mathbb{N}$
        \item $m = 0 \lor n = 0$ - очевидно
        \item $m < 0 \land n > 0$:
        \begin{align*}
            (a^m)^n = \left(\frac{1}{a^{-m}}\right)^n = \frac{1}{a^{-m \cdot n}} = a^{m \cdot n}
        \end{align*}
        \item $m > 0 \land n < 0$:
        \begin{align*}
            (a^m)^n = \frac{1}{(a^m)^{-n}} = \frac{1}{a^{-m \cdot n}} = a^{m \cdot n}
        \end{align*}
        \item $m < 0 \land n < 0$:
        \begin{align*}
            (a^m)^n = \frac{1}{(a^m)^{-n}} \stackrel{3.}{=} \frac{1}{a^{-m \cdot n}} = a^{m \cdot n}
        \end{align*}
    \end{enumerate}
\end{proof}

\begin{theorem}
    $\forall a \in \mathbb{R} \backslash \{0\} \; \forall m, n \in \mathbb{Z}: \; \frac{a^m}{a^n} = a^{m-n}$
\end{theorem}

\begin{proof}
    \hfill

    \begin{enumerate}
        \item $m,n \in \mathbb{N} \land m > n$ - см. доказательство для $\mathbb{N}$
        \item $m,n \in \mathbb{N} \land m = n$ - очевидно (нулевая степень)
        \item $m,n \in \mathbb{N} \land m < n$:
        \begin{align*}
            \frac{a^m}{a^n} = \frac{1}{(\frac{a^n}{a^m})} = \frac{1}{a^{n - m}} = a^{- (n - m)} = a^{m - n}
        \end{align*}
        \item $m = 0 \lor n = 0$ - очевидно
        \item $m < 0 \land n > 0$:
        \begin{align*}
            \frac{a^m}{a^n} = \frac{1}{a^{n} \cdot a^{-m}} = \frac{1}{a^{n - m}} = a^{- (n - m)} = a^{m - n}
        \end{align*}
        \item $m > 0 \land n < 0$:
        \begin{align*}
            \frac{a^m}{a^n} = \frac{m}{(\frac{1}{a^{-n}})} = a^{m} \cdot a^{-n}= a^{m - n}
        \end{align*}
        \item $m < 0 \land n < 0$:
        \begin{align*}
            \frac{a^m}{a^n} = \frac{a^{-n}}{a^{-m}} = a^{-n - (-m)} = a^{m - n}
        \end{align*}
    \end{enumerate}
\end{proof}

\break

\begin{theorem}
    $\forall a, b \in \mathbb{R} \backslash \{0\} \; \forall m \in \mathbb{Z}: \; (ab)^m = a^m \cdot a^n$
\end{theorem}

\begin{proof}
    \hfill

    \begin{enumerate}
        \item $m \in \mathbb{N}$ - см. доказательство для $\mathbb{N}$
        \item $m = 0$ - очевидно
        \item $m \in \mathbb{Z} \land m < 0$:
        \begin{align*}
            (ab)^m = \frac{1}{(ab)^{-m}} = \frac{1}{a^{-m} \cdot b^{-m}} = a^m \cdot a^n
        \end{align*}
    \end{enumerate}
\end{proof}

\section{Арифметический корень натуральной степени}

\begin{definition}[Четная степень]
    Арифметическим корнем натуральной четной степени $n$ из числа $a$ называется число $b$, если оно при возведении в степень $n$ равно числу $a$

    \[
        b =  \sqrt[n]{a} \overset{ n \divisible 2}{\underset{n \in \mathbb{N}}{\iff}} 
        \left\{
        \begin{array}{l}
            b \ge 0 \\
            b^n = a
        \end{array}\right.
    \]
\end{definition}

\begin{definition}[Нечетная степень]
    Арифметическим корнем натуральной нечетной степени $n$ из числа $a$ называется число $b$, если оно при возведении в степень $n$ равно числу $a$

    \[
        b =  \sqrt[n]{a} \overset{ n \notdivisible 2}{\underset{n \in \mathbb{N}}{\iff}} b^n = a
    \]
\end{definition}

\begin{example}
    $\sqrt[5]{625} = 4$; $\sqrt[3]{8} = 2$; $\sqrt[4]{-4} \in \varnothing$;
\end{example}

\label{thm:1.2.2}
\begin{theorem}[Теорема о корректности определения]
    \[
        \forall n \in \mathbb{N}: n \divisible 2 \;\;\; \forall a \ge 0 \;\;\; \exists! \; b \ge 0: b^n = a
    \]
\end{theorem}

\label{thm:1.2.3}
\begin{theorem}[Теорема о корректности определения]
    \[
        \forall n \in \mathbb{N}: n \notdivisible 2 \;\;\; \forall a \in \mathbb{R} \;\;\; \exists! \; b \in \mathbb{R}: b^n = a
    \]
\end{theorem}
\newpage
\section{Свойства корня n-ой степени}

\begin{theorem}
    \[
        \forall n \in \mathbb{N}: n \notdivisible 2 \;\;\; \forall a \in \mathbb{R} \;\;\; \sqrt[n]{a^n} = (\sqrt[n]{a})^n = a
    \]
\end{theorem}

\begin{proof}
    \begin{align*}
        &\text{Пусть } \sqrt[n]{a^n} = x \text{, тогда}\\
        &x^n = a^n \overset{n \notdivisible 2}{\iff} x = a \iff \sqrt[n]{a^n} = a \\
        \\
        &\text{Пусть } (\sqrt[n]{a})^n = y \text{, тогда}\\
        &\sqrt[n]{y} = \sqrt[n]{a} \iff (\sqrt[n]{y})^n = a \iff \left(\sqrt[n]{(\sqrt[n]{a})^n}\right)^n = a \iff (\sqrt[n]{a})^n = a
    \end{align*}
\end{proof}

\begin{theorem}
    \begin{align*}
        \forall n \in \mathbb{N}: n \divisible 2 \;\;\; &\forall a \ge 0 \; (\sqrt[n]{a})^n = a \\
        &\forall a \in \mathbb{R} \; \sqrt[n]{a^n} = |a|
    \end{align*}
\end{theorem}

\begin{proof}
    \begin{align*}
        &\text{Пусть } \sqrt[n]{a^n} = x \text{, тогда}\\
        &\left\{\begin{array}{l}
            x^n = a^n \\
            x \ge 0
        \end{array}\right.
        \overset{n \divisible 2}{\iff} x = |a| \iff 
        \left[\begin{array}{l}
                x = a \\
                x = -a
        \end{array}\right. \iff 
        \left[\begin{array}{l}
            \sqrt[n]{a^n} = a \\
            \sqrt[n]{a^n} = -a
        \end{array}\right. 
        \iff \sqrt[n]{a^n} = |a|\\
        \\
        &\text{Пусть } (\sqrt[n]{a})^n = y \text{, тогда}\\
        &\left\{\begin{array}{l}
            \sqrt[n]{a} = \sqrt[n]{y} \\
            \sqrt[n]{a} \ge 0
        \end{array}\right. \iff
        \left\{\begin{array}{l}
            \left\{\begin{array}{l}
                (\sqrt[n]{y})^n = a \\
                \sqrt[n]{y} \ge 0
            \end{array}\right. \\
            \sqrt[n]{a} \ge 0
        \end{array}\right. \underset{a \ge 0}{\iff}
        \left(\sqrt[n]{(\sqrt[n]{a})^n}\right)^n = a \iff
        (\sqrt[n]{a})^n = |a| \iff \\
        &\iff (\sqrt[n]{a})^n = a
    \end{align*}
\end{proof}

\begin{theorem}
    \begin{align*}
        \forall n \in \mathbb{N}: n \notdivisible 2 \;\;\; \forall a, b \in \mathbb{R}  \;\;\; \sqrt[n]{ab} = \sqrt[n]{a} \cdot \sqrt[n]{b}
    \end{align*}
\end{theorem}

\begin{proof}
    \hfill
    \begin{align*}
        &\text{Пусть } x = \sqrt[n]{a} \land y = \sqrt[n]{b} \text{, тогда}\\
        &\left\{\begin{array}{l}
            x^n = a \\
            y^n = b \\
        \end{array}\right. \implies
        \sqrt[n]{ab} = \sqrt[n]{(xy)^n} \iff
        xy = \sqrt[n]{ab} = \sqrt[n]{a} \cdot \sqrt[n]{b}
    \end{align*}
\end{proof}

\break

\begin{theorem}
    \begin{align*}
        \forall n \in \mathbb{N}: n \divisible 2 \;\;\; \forall a, b \ge 0 \;\;\; \sqrt[n]{ab} = \sqrt[n]{a} \cdot \sqrt[n]{b}
    \end{align*}
\end{theorem}

\begin{proof}
    \hfill
    \begin{align*}
        &\text{Пусть } x = \sqrt[n]{a} \land y = \sqrt[n]{b} \text{, тогда}\\
        &\left\{\begin{array}{l}
            x^n = a \\
            x \ge 0 \\
            y^n = b \\
            y \ge 0
        \end{array}\right. \underset{a, b \ge 0}{\implies}
        \left\{\begin{array}{l}
            x^n = a \\
            y^n = b
        \end{array}\right. \implies
        \sqrt[n]{ab} = \sqrt[n]{(xy)^n} \underset{x, y \ge 0}{\iff}
        xy = \sqrt[n]{ab} = \sqrt[n]{a} \cdot \sqrt[n]{b}
    \end{align*}
\end{proof}

\begin{theorem}
    \begin{align*}
        \forall n \in \mathbb{N}: n \notdivisible 2 \;\;\; \forall a \in \mathbb{R} \;\;\; \forall b \in \mathbb{R} \backslash \{0\} \;\;\; \sqrt[n]{\frac{a}{b}} = \frac{\sqrt[n]{a}}{\sqrt[n]{b}}
    \end{align*}
\end{theorem}

\begin{proof}
    \hfill
    \begin{align*}
        &\text{Пусть } x = \sqrt[n]{a} \land y = \sqrt[n]{b} \text{, тогда}\\
        &\left(\frac{x}{y}\right)^n = \frac{x^n}{y^n} = \frac{a}{b} \iff \frac{x}{y} = \sqrt[n]{\frac{a}{b}} = \frac{\sqrt[n]{a}}{\sqrt[n]{b}}
    \end{align*}
\end{proof}

\begin{theorem}
    \begin{align*}
        \forall n \in \mathbb{N}: n \divisible 2 \;\;\; \forall a \ge 0  \;\;\; \forall b > 0 \;\;\; \sqrt[n]{\frac{a}{b}} = \frac{\sqrt[n]{a}}{\sqrt[n]{b}}
    \end{align*}
\end{theorem}

\begin{proof}
    \hfill
    \begin{align*}
        &\text{Пусть } x = \sqrt[n]{a} \land y = \sqrt[n]{b} \text{, тогда}\\
        &\left\{\begin{array}{l}
            a \ge 0 \\
            b > 0 \\
        \end{array}\right. \implies
        \left\{\begin{array}{l}
            x \ge 0 \\
            y > 0 \\
        \end{array}\right. \implies
        \frac{x}{y} \ge 0 \;\;\; (1)
        \\ \\
        &\left(\frac{x}{y}\right)^n = \frac{x^n}{y^n} = \frac{a}{b} \overset{(1)}{\iff} \frac{x}{y} = \sqrt[n]{\frac{a}{b}} = \frac{\sqrt[n]{a}}{\sqrt[n]{b}}
    \end{align*}
\end{proof}

\begin{theorem}
    \begin{align*}
        \forall m,n \in \mathbb{N}: m,n \notdivisible 2 \;\;\; \forall a \in \mathbb{R} \;\;\; \sqrt[n]{\sqrt[m]{a}} = \sqrt[m \cdot n]{a}
    \end{align*}
\end{theorem}

\begin{proof}
    \hfill
    \begin{align*}
        &\text{Пусть } x = \sqrt[n]{\sqrt[m]{a}} \text{, тогда} \\
        &x^n = \sqrt[m]{a} \iff
        x^{n \cdot m} = a \iff
        x = \sqrt[m \cdot n]{a} = \sqrt[n]{\sqrt[m]{a}}
    \end{align*}
\end{proof}

\begin{theorem}
    \begin{align*}
        \forall m,n \in \mathbb{N}: m,n \divisible 2 \;\;\; \forall a \ge 0 \;\;\; \sqrt[n]{\sqrt[m]{a}} = \sqrt[m \cdot n]{a}
    \end{align*}
\end{theorem}

\begin{proof}
    \hfill
    \begin{align*}
        &\text{Пусть } x = \sqrt[n]{\sqrt[m]{a}} \text{, тогда} \\
        &\left\{\begin{array}{l}
            x^n = \sqrt[m]{a} \\
            x \ge 0 \\
        \end{array}\right. \underset{a \ge 0}{\iff}
        x^{n \cdot m} = a \underset{a \ge 0}{\iff}
        x = \sqrt[m \cdot n]{a} = \sqrt[n]{\sqrt[m]{a}}
    \end{align*}
\end{proof}

\begin{theorem}
    \begin{align*}
        \forall m,n \in \mathbb{N}: m,n \notdivisible 2 \;\;\; \forall a \in \mathbb{R} \;\;\; \sqrt[n]{a} = \sqrt[n \cdot m]{a^m}
    \end{align*}
\end{theorem}

\begin{proof}
    \hfill
    \begin{align*}
        &\text{Пусть } x = \sqrt[n]{a} \text{, тогда} \\
        & x^n = a \iff
        x^{n \cdot m} = a^m \iff
        x = \sqrt[n \cdot m]{a^m} = \sqrt[n]{a}
    \end{align*}
\end{proof}

\begin{theorem}
    \begin{align*}
        \forall m,n \in \mathbb{N}: m,n \divisible 2 \;\;\; \forall a \ge 0 \;\;\; \sqrt[n]{a} = \sqrt[n \cdot m]{a^m}
    \end{align*}
\end{theorem}

\begin{proof}
    \hfill
    \begin{align*}
        &\text{Пусть } x = \sqrt[n]{a} \text{, тогда} \\
        &\left\{\begin{array}{l}
            x^n = a \\
            x \ge 0 \\
        \end{array}\right. \underset{a \ge 0}{\iff}
        x^{n \cdot m} = a^m \underset{a \ge 0}{\iff}
        x = \sqrt[n \cdot m]{a^m} = \sqrt[n]{a}
    \end{align*}
\end{proof}

\begin{theorem}
    \begin{align*}
        \forall n \in \mathbb{N}: n \notdivisible 2 \;\;\; \forall m \in \mathbb{Z}: m \notdivisible 2 \;\;\; \forall a \in \mathbb{R} \backslash \{0\} \;\;\; 
        \sqrt[n]{a^m} = (\sqrt[n]{a})^m
    \end{align*}
\end{theorem}

\begin{proof}
    \hfill
    \begin{align*}
        &\text{Пусть } x = (\sqrt[n]{a})^m \text{, тогда} \\
        &x^n = ((\sqrt[n]{a})^m)^n \iff
        x^n = ((\sqrt[n]{a})^n)^m = a^m \iff
        x = \sqrt[n]{a^m} = (\sqrt[n]{a})^m
    \end{align*}
\end{proof}

\begin{theorem}
    \begin{align*}
        \forall n \in \mathbb{N}: n \divisible 2 \;\;\; \forall m \in \mathbb{Z}: m \divisible 2 \;\;\; \forall a > 0 \;\;\; 
        \sqrt[n]{a^m} = (\sqrt[n]{a})^m
    \end{align*}
\end{theorem}

\begin{proof}
    \hfill
    \begin{align*}
        &\text{Пусть } x = (\sqrt[n]{a})^m \text{, тогда} \\
        &x^n = ((\sqrt[n]{a})^m)^n \iff
        x^n = ((\sqrt[n]{a})^n)^m = a^m \overset{m \divisible 2}{\underset{a \neq 0}{\implies}}
        \left\{\begin{array}{l}
            x^n = a^m \\
            x > 0 \\
        \end{array}\right. \iff
        x = \sqrt[n]{a^m} = (\sqrt[n]{a})^m
    \end{align*}
\end{proof}

\begin{theorem}
    \begin{align*}
        \forall n \in \mathbb{N}: n \notdivisible 2 \;\;\; \forall a,b \in \mathbb{R}  \;\;\; a \le b \iff \sqrt[n]{a} \le \sqrt[n]{b}
    \end{align*}
\end{theorem}

\begin{proof}
    \hfill \\

    Пусть $x = \sqrt[n]{a} \land y = \sqrt[n]{b}$, тогда
    \begin{enumerate}
        \item $x = 0 \lor y = 0$ - очевидно
        \item $sign(x) \neq sign(y)$ - очевидно
        \item $x > 0 \land y > 0$:
        \begin{align*}
            &a-b = x^n - y^n = (x-y)\overbrace{(x^{n-1} + x^{n-2}y + x^{n-3}y^2 + \dots + y^{n-1})}^{> 0, \text{ так как } x \text{ и } y \text{ больше нуля}}
            \implies sign(a-b) = sign(x-y) \\
            &\implies sign(a-b) = sign(\sqrt[n]{a} - \sqrt[n]{b}) \\
        \end{align*}
        \item $x < 0 \land y < 0$:
        \begin{align*}
            &a-b = x^n - y^n = (x-y)\overbrace{(x^{n-1} + x^{n-2}y + x^{n-3}y^2 + \dots + y^{n-1})}^{> 0, \text{ *}}
            \implies sign(a-b) = sign(x-y) \\
            &\implies sign(a-b) = sign(\sqrt[n]{a} - \sqrt[n]{b}) \\
        \end{align*}
        \text{* в каждом слагаемом все множетили возводятся в степени одинаковой четности.} \\
        \text{Если степень четная, то такое слагаемое будет положительным, если нечетное то тогда} \\
        \text{слагаемое будет произведением двух отрицательных чисел и тоже будет положительным.}
    \end{enumerate}
\end{proof}

\begin{theorem}
    \begin{align*}
        \forall n \in \mathbb{N}: n \divisible 2 \;\;\; \forall a,b \ge 0 \;\;\; a \le b \iff \sqrt[n]{a} \le \sqrt[n]{b}
    \end{align*}
\end{theorem}

\begin{proof}
    \hfill \\

    Пусть $x = \sqrt[n]{a} \land y = \sqrt[n]{b}$, тогда
    \begin{enumerate}
        \item $x = 0 \lor y = 0$ - очевидно
        \item $x > 0 \land y > 0$:
        \begin{align*}
            &a-b = x^n - y^n = (x-y)\overbrace{(x^{n-1} + x^{n-2}y + x^{n-3}y^2 + \dots + y^{n-1})}^{> 0, \text{ так как } x \text{ и } y \text{ больше нуля}}
            \implies sign(a-b) = sign(x-y) \\
            &\implies sign(a-b) = sign(\sqrt[n]{a} - \sqrt[n]{b}) \\
        \end{align*}
    \end{enumerate}
\end{proof}

\newpage
\section{Степень с рациональным показателем}

\begin{remind}
$\mathbb{Q} = \{x \in \mathbb{R} \; | \; \exists m \in \mathbb{Z}, n \in \mathbb{N} \; : \frac{m}{n} = x \}$
\end{remind}

\begin{definition}
    \begin{align*}
        &\forall r \in \mathbb{Q} \; : \; r = \frac{m}{n} \text{, где } m \in \mathbb{Z} \land n \in \mathbb{N} \\
        &a^r = \sqrt[n]{a^m}
    \end{align*}
\end{definition}

\begin{example}
    $16^{\frac{3}{4}} = \sqrt[4]{16^3} = 8$ 
\end{example}

\begin{example}
    $27^{\frac{4}{3}} = 81$
\end{example}

\begin{example}
    $1000^{-\frac{2}{3}} = 0,01$
\end{example}

\begin{theorem}[О корректности определения]
    Для любого $a > 0$ и любого $r \in \mathbb{Q}$ значение $a^r$ существует и не зависит от выбора представления числа $r$ в виде дроби.
\end{theorem}

\begin{proof}[Доказательство существования]
    Следует из теорем \hyperref[thm:1.2.2]{о корректности определеня корня n-ой степени}
\end{proof}

\begin{proof}[Доказательство единственности]
    \begin{align*}
        &\text{Предположим, что } r = \frac{m}{n} = \frac{s}{t} \text{, где } m,s \in \mathbb{Z} \land n,t \in \mathbb{N} \\
        &\text{Тогда } \exists \; k \in \mathbb{Z}, \; l,p,q \in \mathbb{N} \text{ такие, что } \\
        &\left\{\begin{array}{l}
            \frac{m}{n} = \frac{k}{l} = \frac{s}{t} \\
            \gcd(k,l) = 1 \\
            m = p \cdot k \\
            n = p \cdot l \\
            s = q \cdot k \\
            t = q \cdot l
        \end{array}\right. \\
        &\text{Тогда } \\
        &a^{\frac{m}{n}} = \sqrt[n]{a^m} = \sqrt[p \cdot l]{a^{p \cdot k}} = \sqrt[l]{a^k} \\
        &a^{\frac{s}{t}} = \sqrt[t]{a^s} = \sqrt[q \cdot l]{a^{q \cdot k}} = \sqrt[l]{a^k}
    \end{align*}
\end{proof}

\begin{remark}
    Ноль можно возводить в рациональную положительную степень и нельзя в отрицательную
\end{remark}

\begin{remark}
    Отрицательное число нельзя возводить в рациональную степень, даже если формально, по формуле результат может быть получен.
\end{remark}

\begin{example}
    $(-1)^\frac{1}{3} = \sqrt[3]{-1} = -1$
\end{example}

\begin{example}
    $(-1)^\frac{2}{6} = \sqrt[6]{(-1)^2} = 1$ (но ведь можно первым шагом сократить, тогда получим $\sqrt[3]{-1}$ что уже равно $-1$ ?!)
\end{example}
\newpage
\section{Свойства степени с рациональным показателем}

\begin{theorem}
    \begin{align*}
        &\forall a > 0 \;\; \forall r, q \in \mathbb{Q} \\
        &a^r \cdot a^q = a^{r + q}
    \end{align*}
\end{theorem}

\begin{proof}
    \begin{align*}
        &r, q \in \mathbb{Q} \iff
        \left\{\begin{array}{l}
            m,s \in \mathbb{Z} \\
            n,t \in \mathbb{N} \\
            r = \frac{m}{n} \\
            q = \frac{s}{t}
        \end{array}\right. \Rightarrow
        a^r \cdot a^q = a^\frac{m}{n} \cdot a^\frac{s}{t} = \sqrt[n]{a^m} \cdot \sqrt[t]{a^s} = \sqrt[nt]{a^{mt}} \cdot \sqrt[nt]{a^{sn}} = \\
        &= \sqrt[nt]{a^{mt} \cdot a^{sn}} = \sqrt[nt]{a^{mt + sn}}
        = a^\frac{mt + sn}{nt} = a^{\frac{m}{m} + \frac{s}{t}} = a^{r + q}
    \end{align*}
\end{proof}

\begin{theorem}
    \begin{align*}
        &\forall a > 0 \;\; \forall r, q \in \mathbb{Q} \\
        &a^r : a^q = a^{r - q}
    \end{align*}
\end{theorem}

\begin{proof}
    \begin{align*}
        &r, q \in \mathbb{Q} \iff
        \left\{\begin{array}{l}
            m,s \in \mathbb{Z} \\
            n,t \in \mathbb{N} \\
            r = \frac{m}{n} \\
            q = \frac{s}{t}
        \end{array}\right. \Rightarrow
        a^r : a^q = \frac{a^\frac{m}{n}}{a^\frac{s}{t}} = \frac{\sqrt[n]{a^m}}{\sqrt[t]{a^s}} = \frac{\sqrt[nt]{a^{mt}}}{\sqrt[nt]{a^{sn}}}
        = \sqrt[nt]{\frac{a^{mt}}{a^{sn}}} = \sqrt[nt]{a^{mt - sn}} = a^\frac{mt - sn}{nt} = \\
        &= a^{\frac{m}{n} + \frac{s}{t}} = a^{r - q}
    \end{align*}
\end{proof}
	\chapter{Графики и функции}

\section{Обратная функция}

\begin{remind}
    $f: A \rightarrow B$ иньекция, если $\forall x_1, x_2 \in A \;\; x_1 \neq x_2 \iff f(x_1) \neq f(x_2)$
\end{remind}

\begin{definition}[Сюръекция]
    $f: A \rightarrow B$ сюръекция, если $\forall y \in B \; \exists x \in A \; : \; y = f(x)$
\end{definition}

\begin{definition}[Биекция]
    $f: A \rightarrow B$ биекция, если она и иньекция и сюръекция одновременно.
\end{definition}

\begin{remark}
    Биекцию так же называют \textbf{взаимно однозначным соответствием}.
\end{remark}

\begin{definition}[Обратная функция]
    Функция $g: B \rightarrow A$ называется обратной к $f: A \rightarrow B$
    если выполняется условие:
    \begin{align*}
        \left\{\begin{array}{l}
            \forall x \in A \;\;\; g(f(x)) = x \\
            \forall y \in B \;\;\; f(g(y)) = y
        \end{array}\right.
    \end{align*}
\end{definition}

\begin{remark}
    $f: A \rightarrow B$ обратима тогда и только тогда когда биекция.
\end{remark}

\begin{example}
    $y=x$ обратная сама к себе
\end{example}

\begin{example}
    $y=2x + 1 \iff x = \frac{y - 1}{2}$ Тогда обратная функция к функции $f(x) = 2x + 1$ будет функция $g(x) = \frac{y - 1}{2}$
\end{example}

\begin{example}
    $f(x) = x^2 - 2x + 3$, при $A = \left(-\infty; 1\right]$
    \begin{align*}
        y = x^2 - 2x + 3 \iff x = 1 \pm \sqrt{y - 2} \overset{x \in A}{\iff} x = 1 - \sqrt{y - 2}
    \end{align*}
    Тогда $g(x) = 1 - \sqrt{y - 2}$ обартная к $f$ на множестве $A$
\end{example}

\begin{definition}
    Функция, обратная к функции $f(x)$ обозначается как $f(x)^{-1}$ 
\end{definition}

\begin{theorem}[О графике обратной функции]
    Графики прямой и обартной функции симметричны относительно прямой $y = x$.
\end{theorem}

\begin{proof}
    \begin{align*}
        &\forall (x; y) \; \forall f: A \rightarrow B, \; g: B \rightarrow A \;\;\; (x; y) \in \Gamma_f \iff \\
        &\iff  \left\{\begin{array}{l}
            x \in D(f) \\
            y = f(x)
        \end{array}\right. \iff
        \left\{\begin{array}{l}
            y \in D(g) \\
            g(y) = x 
        \end{array}\right. \iff
        (x; y) \in \Gamma_g
    \end{align*}

    Если $y = f(x)$ то $y \in D(g)$, ведь функция $f(x)$ как бы <<возвращает>> множество $B$ которое для функции $g$ является областью определения. \\
    А зная, что $g(x) = f(x)^{-1}$ мы можем применить функцию к обоим частям уравнения и получим в правой части <<чистый аргумент>> без функции.
\end{proof}

\break

\begin{theorem}[О монотонности обратной функции]
    \begin{align*}
        &\text{Если } f: A \rightarrow B, \; g: B \rightarrow A \land g(x) = f(x)^{-1} \text{ то} \\
        &\left\{\begin{array}{l}
            f \uparrow A \Rightarrow g \uparrow B \\
            f \downarrow A \Rightarrow g \downarrow B
        \end{array}\right.
    \end{align*}
\end{theorem}

\begin{proof}
    \begin{align*}
        1) & \; \forall x_1, x_2 \in A \;\; x_1 > x_2 \Rightarrow f(x_1) > f(x_2)  \\
        &\text{Предположим, } \exists y_1, y_2 \in B \;\; y_1 > y_2 \land g(x_1) \le g(x_2) \\
        &\left\{\begin{array}{l}
            g(y_1) = x_1 \\
            g(y_2) = x_2
        \end{array}\right. \Rightarrow
        x_1 \le x_2 \Rightarrow f(x_1) \le f(x_2) \; ?! \\ \\
        2) & \; \forall x_1, x_2 \in A \;\; x_1 > x_2 \Rightarrow f(x_1) < f(x_2)  \\
        &\text{Предположим, } \exists y_1, y_2 \in B \;\; y_1 > y_2 \land g(x_1) \ge g(x_2) \\
        &\left\{\begin{array}{l}
            g(y_1) = x_1 \\
            g(y_2) = x_2
        \end{array}\right. \Rightarrow
        x_1 \ge x_2 \Rightarrow f(x_1) \ge f(x_2) \; ?!
    \end{align*}
\end{proof}


\section{Иррациональные неравенства}

\begin{align*}
    \begin{array}{c}
        a, b \in A \\
        f \uparrow A
    \end{array}
    \hspace{1.5cm}
    a \le b \iff f(a) \le f(b) \\ \\
    \begin{array}{c}
        a, b \in A \\
        f \downarrow A
    \end{array}
    \hspace{1.5cm}
    a \le b \iff f(a) \ge f(b) \\
\end{align*}

\begin{remark}
    Возведение обоих частей неравенства в нечетную положительную степень всегда является равносильным преобразованием.
\end{remark}

\begin{remark}
    Если обе части неравенства неотрицательны, то их можно возводить в четную натуральную степень, но возможно потребуются ограничения по ОДЗ.
\end{remark}

\begin{example}
    $\sqrt{x - 2} > 3 \iff x - 2 > 9 \iff x > 11$
\end{example}

\begin{example}
    \begin{align*}
        &\sqrt{x - 2} \le 3 \iff
        \left\{\begin{array}{l}
            x - 2 \ge 0 \\
            x - 2  \le 9
        \end{array}\right. \iff
        \left\{\begin{array}{l}
            x \ge 2 \\
            x \le 11
        \end{array}\right. \iff
        x \in [2; 11]
    \end{align*}
\end{example}

\begin{remark}
    Если обе части неравенства неположительные числа, то неравенство можно возвести в четную натуральную степень, 
    но знак неравенства поменяется на противположгный
\end{remark}

\begin{remark}
    Если обе части неравенства имеют разный знак, то их нельзя возводить в четную натуральную степень, но при этом о выполнении неравенства
    можно непосредственно судить.
\end{remark}

\begin{example}
    \begin{align*}
        &\sqrt{x - 2} \le -x \iff
        \left\{\begin{array}{l}
            x \le 0 \\
            x - 2 \le x^2 \\
            x + 2 \ge 0
        \end{array}\right. \iff
        \left\{\begin{array}{l}
            x \in [-2; 0] \\
            (x + 1)(x - 2) \ge 0
        \end{array}\right. \iff
        \left\{\begin{array}{l}
            x \in [-2; 0] \\
            x \in (-\infty; -1] \cup [2; +\infty) \ge 0
        \end{array}\right. \\
        &\text{Ответ: } [-2; -1]
    \end{align*}
\end{example}

\begin{example}
    \begin{align*}
        &\sqrt{x - 2} \ge -x \iff
        \left[
            \begin{array}{l}
                \left\{\begin{array}{l}
                    x > 0 \\
                    x \ge -2
                \end{array}\right. \\
                \left\{\begin{array}{l}
                    x \le 0 \\
                    x + 2 \ge x^2
                \end{array}\right.
            \end{array}
        \right. \iff
        \left[
            \begin{array}{l}
                x \ge -2 \\
                \left\{\begin{array}{l}
                    x \le 0 \\
                    (x + 1)(x - 2) \le 0
                \end{array}\right.
            \end{array}
        \right. \iff
        \left[
            \begin{array}{l}
                x \ge -2 \\
                \left\{\begin{array}{l}
                    x \le 0 \\
                    x \in [-1; 2]
                \end{array}\right.
            \end{array}
        \right. \\
        &\text{Ответ: } [-1; +\infty)
    \end{align*}
\end{example}


\end{document}

