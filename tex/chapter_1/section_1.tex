\section{Степень с целым показателем}

\begin{definition}
    \[
        m \in \mathbb{Z}: \; a^m = 
        \left\{
        \begin{array}{l}
            a^m, \; m \in \mathbb{N} \\ \\
            1, \; m = 0 \\ \\
            \frac{1}{a^m}, \; m \in  \mathbb{Z} \land n < 0 \\
        \end{array}\right.
    \]
\end{definition}

\begin{remark}
    $a^m$, при $m \in \mathbb{Z} \; \backslash \; \mathbb{N}$ не определенно при $a = 0$. Иными словами, ноль нельзя возводить в неположительную степень.
\end{remark}

\begin{example}
    $2^{-3} = \frac{1}{2^3} = \frac{1}{8}$; $(\frac{3}{2})^{-2} = \frac{2^2}{3^2} = \frac{4}{9}$;
\end{example}

\begin{theorem}
    $\forall a \in \mathbb{R} \backslash \{0\} \; \forall m, n \in \mathbb{Z}: \; a^m \cdot a^n = a^{m + n}$
\end{theorem}

\begin{proof}
    \hfill

    \begin{enumerate}
        \item $m,n \in \mathbb{N}$ - см. доказательство для $\mathbb{N}$
        \item $m = 0 \lor n = 0$ - очевидно
        \item $m < 0 \land n > 0$:
        \begin{align*}
            a^m \cdot a^n = \frac{a^n}{a^{-m}} = a^{n - (-m)} = a^{m + n}
        \end{align*}
        \item $m > 0 \land n < 0$:
        \begin{align*}
            a^m \cdot a^n = \frac{a^m}{a^{-n}} = a^{m - (-n)} = a^{m + n}
        \end{align*}
        \item $m < 0 \land n < 0$:
        \begin{align*}
            a^m \cdot a^n = \frac{1}{a^{-n} \cdot a^{-m}} = \frac{1}{a^{-(n+m)}} = a^{m + n}
        \end{align*}
    \end{enumerate}
\end{proof}

\break

\begin{theorem}
    $\forall a \in \mathbb{R} \backslash \{0\} \; \forall m, n \in \mathbb{Z}: \; (a^m)^n = a^{m \cdot n}$
\end{theorem}

\begin{proof}
    \hfill

    \begin{enumerate}
        \item $m,n \in \mathbb{N}$ - см. доказательство для $\mathbb{N}$
        \item $m = 0 \lor n = 0$ - очевидно
        \item $m < 0 \land n > 0$:
        \begin{align*}
            (a^m)^n = \left(\frac{1}{a^{-m}}\right)^n = \frac{1}{a^{-m \cdot n}} = a^{m \cdot n}
        \end{align*}
        \item $m > 0 \land n < 0$:
        \begin{align*}
            (a^m)^n = \frac{1}{(a^m)^{-n}} = \frac{1}{a^{-m \cdot n}} = a^{m \cdot n}
        \end{align*}
        \item $m < 0 \land n < 0$:
        \begin{align*}
            (a^m)^n = \frac{1}{(a^m)^{-n}} \stackrel{3.}{=} \frac{1}{a^{-m \cdot n}} = a^{m \cdot n}
        \end{align*}
    \end{enumerate}
\end{proof}

\begin{theorem}
    $\forall a \in \mathbb{R} \backslash \{0\} \; \forall m, n \in \mathbb{Z}: \; \frac{a^m}{a^n} = a^{m-n}$
\end{theorem}

\begin{proof}
    \hfill

    \begin{enumerate}
        \item $m,n \in \mathbb{N} \land m > n$ - см. доказательство для $\mathbb{N}$
        \item $m,n \in \mathbb{N} \land m = n$ - очевидно (нулевая степень)
        \item $m,n \in \mathbb{N} \land m < n$:
        \begin{align*}
            \frac{a^m}{a^n} = \frac{1}{(\frac{a^n}{a^m})} = \frac{1}{a^{n - m}} = a^{- (n - m)} = a^{m - n}
        \end{align*}
        \item $m = 0 \lor n = 0$ - очевидно
        \item $m < 0 \land n > 0$:
        \begin{align*}
            \frac{a^m}{a^n} = \frac{1}{a^{n} \cdot a^{-m}} = \frac{1}{a^{n - m}} = a^{- (n - m)} = a^{m - n}
        \end{align*}
        \item $m > 0 \land n < 0$:
        \begin{align*}
            \frac{a^m}{a^n} = \frac{m}{(\frac{1}{a^{-n}})} = a^{m} \cdot a^{-n}= a^{m - n}
        \end{align*}
        \item $m < 0 \land n < 0$:
        \begin{align*}
            \frac{a^m}{a^n} = \frac{a^{-n}}{a^{-m}} = a^{-n - (-m)} = a^{m - n}
        \end{align*}
    \end{enumerate}
\end{proof}

\break

\begin{theorem}
    $\forall a, b \in \mathbb{R} \backslash \{0\} \; \forall m \in \mathbb{Z}: \; (ab)^m = a^m \cdot a^n$
\end{theorem}

\begin{proof}
    \hfill

    \begin{enumerate}
        \item $m \in \mathbb{N}$ - см. доказательство для $\mathbb{N}$
        \item $m = 0$ - очевидно
        \item $m \in \mathbb{Z} \land m < 0$:
        \begin{align*}
            (ab)^m = \frac{1}{(ab)^{-m}} = \frac{1}{a^{-m} \cdot b^{-m}} = a^m \cdot a^n
        \end{align*}
    \end{enumerate}
\end{proof}
