\section{Степень с рациональным показателем}

\begin{remind}
$\mathbb{Q} = \{x \in \mathbb{R} \; | \; \exists m \in \mathbb{Z}, n \in \mathbb{N} \; : \frac{m}{n} = x \}$
\end{remind}

\begin{definition}
    \begin{align*}
        &\forall r \in \mathbb{Q} \; : \; r = \frac{m}{n} \text{, где } m \in \mathbb{Z} \land n \in \mathbb{N} \\
        &a^r = \sqrt[n]{a^m}
    \end{align*}
\end{definition}

\begin{example}
    $16^{\frac{3}{4}} = \sqrt[4]{16^3} = 8$ 
\end{example}

\begin{example}
    $27^{\frac{4}{3}} = 81$
\end{example}

\begin{example}
    $1000^{-\frac{2}{3}} = 0,01$
\end{example}

\begin{theorem}[О корректности определения]
    Для любого $a > 0$ и любого $r \in \mathbb{Q}$ значение $a^r$ существует и не зависит от выбора представления числа $r$ в виде дроби.
\end{theorem}

\begin{proof}[Доказательство существования]
    Следует из теорем \hyperref[thm:1.2.2]{о корректности определеня корня n-ой степени}
\end{proof}

\begin{proof}[Доказательство единственности]
    \begin{align*}
        &\text{Предположим, что } r = \frac{m}{n} = \frac{s}{t} \text{, где } m,s \in \mathbb{Z} \land n,t \in \mathbb{N} \\
        &\text{Тогда } \exists \; k \in \mathbb{Z}, \; l,p,q \in \mathbb{N} \text{ такие, что } \\
        &\left\{\begin{array}{l}
            \frac{m}{n} = \frac{k}{l} = \frac{s}{t} \\
            \gcd(k,l) = 1 \\
            m = p \cdot k \\
            n = p \cdot l \\
            s = q \cdot k \\
            t = q \cdot l
        \end{array}\right. \\
        &\text{Тогда } \\
        &a^{\frac{m}{n}} = \sqrt[n]{a^m} = \sqrt[p \cdot l]{a^{p \cdot k}} = \sqrt[l]{a^k} \\
        &a^{\frac{s}{t}} = \sqrt[t]{a^s} = \sqrt[q \cdot l]{a^{q \cdot k}} = \sqrt[l]{a^k}
    \end{align*}
\end{proof}

\begin{remark}
    Ноль можно возводить в рациональную положительную степень и нельзя в отрицательную
\end{remark}

\begin{remark}
    Отрицательное число нельзя возводить в рациональную степень, даже если формально, по формуле результат может быть получен.
\end{remark}

\begin{example}
    $(-1)^\frac{1}{3} = \sqrt[3]{-1} = -1$
\end{example}

\begin{example}
    $(-1)^\frac{2}{6} = \sqrt[6]{(-1)^2} = 1$ (но ведь можно первым шагом сократить, тогда получим $\sqrt[3]{-1}$ что уже равно $-1$ ?!)
\end{example}