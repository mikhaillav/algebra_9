\section{Свойства корня n-ой степени}

\begin{theorem}
    \[
        \forall n \in \mathbb{N}: n \notdivisible 2 \;\;\; \forall a \in \mathbb{R} \;\;\; \sqrt[n]{a^n} = (\sqrt[n]{a})^n = a
    \]
\end{theorem}

\begin{proof}
    \begin{align*}
        &\text{Пусть } \sqrt[n]{a^n} = x \text{, тогда}\\
        &x^n = a^n \overset{n \notdivisible 2}{\iff} x = a \iff \sqrt[n]{a^n} = a \\
        \\
        &\text{Пусть } (\sqrt[n]{a})^n = y \text{, тогда}\\
        &\sqrt[n]{y} = \sqrt[n]{a} \iff (\sqrt[n]{y})^n = a \iff \left(\sqrt[n]{(\sqrt[n]{a})^n}\right)^n = a \iff (\sqrt[n]{a})^n = a
    \end{align*}
\end{proof}

\begin{theorem}
    \begin{align*}
        \forall n \in \mathbb{N}: n \divisible 2 \;\;\; &\forall a \ge 0 \; (\sqrt[n]{a})^n = a \\
        &\forall a \in \mathbb{R} \; \sqrt[n]{a^n} = |a|
    \end{align*}
\end{theorem}

\begin{proof}
    \begin{align*}
        &\text{Пусть } \sqrt[n]{a^n} = x \text{, тогда}\\
        &\left\{\begin{array}{l}
            x^n = a^n \\
            x \ge 0
        \end{array}\right.
        \overset{n \divisible 2}{\iff} x = |a| \iff 
        \left[\begin{array}{l}
                x = a \\
                x = -a
        \end{array}\right. \iff 
        \left[\begin{array}{l}
            \sqrt[n]{a^n} = a \\
            \sqrt[n]{a^n} = -a
        \end{array}\right. 
        \iff \sqrt[n]{a^n} = |a|\\
        \\
        &\text{Пусть } (\sqrt[n]{a})^n = y \text{, тогда}\\
        &\left\{\begin{array}{l}
            \sqrt[n]{a} = \sqrt[n]{y} \\
            \sqrt[n]{a} \ge 0
        \end{array}\right. \iff
        \left\{\begin{array}{l}
            \left\{\begin{array}{l}
                (\sqrt[n]{y})^n = a \\
                \sqrt[n]{y} \ge 0
            \end{array}\right. \\
            \sqrt[n]{a} \ge 0
        \end{array}\right. \underset{a \ge 0}{\iff}
        \left(\sqrt[n]{(\sqrt[n]{a})^n}\right)^n = a \iff
        (\sqrt[n]{a})^n = |a| \iff \\
        &\iff (\sqrt[n]{a})^n = a
    \end{align*}
\end{proof}

\begin{theorem}
    \begin{align*}
        \forall n \in \mathbb{N}: n \notdivisible 2 \;\;\; \forall a, b \in \mathbb{R}  \;\;\; \sqrt[n]{ab} = \sqrt[n]{a} \cdot \sqrt[n]{b}
    \end{align*}
\end{theorem}

\begin{proof}
    \hfill
    \begin{align*}
        &\text{Пусть } x = \sqrt[n]{a} \land y = \sqrt[n]{b} \text{, тогда}\\
        &\left\{\begin{array}{l}
            x^n = a \\
            y^n = b \\
        \end{array}\right. \implies
        \sqrt[n]{ab} = \sqrt[n]{(xy)^n} \iff
        xy = \sqrt[n]{ab} = \sqrt[n]{a} \cdot \sqrt[n]{b}
    \end{align*}
\end{proof}

\break

\begin{theorem}
    \begin{align*}
        \forall n \in \mathbb{N}: n \divisible 2 \;\;\; \forall a, b \ge 0 \;\;\; \sqrt[n]{ab} = \sqrt[n]{a} \cdot \sqrt[n]{b}
    \end{align*}
\end{theorem}

\begin{proof}
    \hfill
    \begin{align*}
        &\text{Пусть } x = \sqrt[n]{a} \land y = \sqrt[n]{b} \text{, тогда}\\
        &\left\{\begin{array}{l}
            x^n = a \\
            x \ge 0 \\
            y^n = b \\
            y \ge 0
        \end{array}\right. \underset{a, b \ge 0}{\implies}
        \left\{\begin{array}{l}
            x^n = a \\
            y^n = b
        \end{array}\right. \implies
        \sqrt[n]{ab} = \sqrt[n]{(xy)^n} \underset{x, y \ge 0}{\iff}
        xy = \sqrt[n]{ab} = \sqrt[n]{a} \cdot \sqrt[n]{b}
    \end{align*}
\end{proof}

\begin{theorem}
    \begin{align*}
        \forall n \in \mathbb{N}: n \notdivisible 2 \;\;\; \forall a \in \mathbb{R} \;\;\; \forall b \in \mathbb{R} \backslash \{0\} \;\;\; \sqrt[n]{\frac{a}{b}} = \frac{\sqrt[n]{a}}{\sqrt[n]{b}}
    \end{align*}
\end{theorem}

\begin{proof}
    \hfill
    \begin{align*}
        &\text{Пусть } x = \sqrt[n]{a} \land y = \sqrt[n]{b} \text{, тогда}\\
        &\left(\frac{x}{y}\right)^n = \frac{x^n}{y^n} = \frac{a}{b} \iff \frac{x}{y} = \sqrt[n]{\frac{a}{b}} = \frac{\sqrt[n]{a}}{\sqrt[n]{b}}
    \end{align*}
\end{proof}

\begin{theorem}
    \begin{align*}
        \forall n \in \mathbb{N}: n \divisible 2 \;\;\; \forall a \ge 0  \;\;\; \forall b > 0 \;\;\; \sqrt[n]{\frac{a}{b}} = \frac{\sqrt[n]{a}}{\sqrt[n]{b}}
    \end{align*}
\end{theorem}

\begin{proof}
    \hfill
    \begin{align*}
        &\text{Пусть } x = \sqrt[n]{a} \land y = \sqrt[n]{b} \text{, тогда}\\
        &\left\{\begin{array}{l}
            a \ge 0 \\
            b > 0 \\
        \end{array}\right. \implies
        \left\{\begin{array}{l}
            x \ge 0 \\
            y > 0 \\
        \end{array}\right. \implies
        \frac{x}{y} \ge 0 \;\;\; (1)
        \\ \\
        &\left(\frac{x}{y}\right)^n = \frac{x^n}{y^n} = \frac{a}{b} \overset{(1)}{\iff} \frac{x}{y} = \sqrt[n]{\frac{a}{b}} = \frac{\sqrt[n]{a}}{\sqrt[n]{b}}
    \end{align*}
\end{proof}

\begin{theorem}
    \begin{align*}
        \forall m,n \in \mathbb{N}: m,n \notdivisible 2 \;\;\; \forall a \in \mathbb{R} \;\;\; \sqrt[n]{\sqrt[m]{a}} = \sqrt[m \cdot n]{a}
    \end{align*}
\end{theorem}

\begin{proof}
    \hfill
    \begin{align*}
        &\text{Пусть } x = \sqrt[n]{\sqrt[m]{a}} \text{, тогда} \\
        &x^n = \sqrt[m]{a} \iff
        x^{n \cdot m} = a \iff
        x = \sqrt[m \cdot n]{a} = \sqrt[n]{\sqrt[m]{a}}
    \end{align*}
\end{proof}

\begin{theorem}
    \begin{align*}
        \forall m,n \in \mathbb{N}: m,n \divisible 2 \;\;\; \forall a \ge 0 \;\;\; \sqrt[n]{\sqrt[m]{a}} = \sqrt[m \cdot n]{a}
    \end{align*}
\end{theorem}

\begin{proof}
    \hfill
    \begin{align*}
        &\text{Пусть } x = \sqrt[n]{\sqrt[m]{a}} \text{, тогда} \\
        &\left\{\begin{array}{l}
            x^n = \sqrt[m]{a} \\
            x \ge 0 \\
        \end{array}\right. \underset{a \ge 0}{\iff}
        x^{n \cdot m} = a \underset{a \ge 0}{\iff}
        x = \sqrt[m \cdot n]{a} = \sqrt[n]{\sqrt[m]{a}}
    \end{align*}
\end{proof}

\begin{theorem}
    \begin{align*}
        \forall m,n \in \mathbb{N}: m,n \notdivisible 2 \;\;\; \forall a \in \mathbb{R} \;\;\; \sqrt[n]{a} = \sqrt[n \cdot m]{a^m}
    \end{align*}
\end{theorem}

\begin{proof}
    \hfill
    \begin{align*}
        &\text{Пусть } x = \sqrt[n]{a} \text{, тогда} \\
        & x^n = a \iff
        x^{n \cdot m} = a^m \iff
        x = \sqrt[n \cdot m]{a^m} = \sqrt[n]{a}
    \end{align*}
\end{proof}

\begin{theorem}
    \begin{align*}
        \forall m,n \in \mathbb{N}: m,n \divisible 2 \;\;\; \forall a \ge 0 \;\;\; \sqrt[n]{a} = \sqrt[n \cdot m]{a^m}
    \end{align*}
\end{theorem}

\begin{proof}
    \hfill
    \begin{align*}
        &\text{Пусть } x = \sqrt[n]{a} \text{, тогда} \\
        &\left\{\begin{array}{l}
            x^n = a \\
            x \ge 0 \\
        \end{array}\right. \underset{a \ge 0}{\iff}
        x^{n \cdot m} = a^m \underset{a \ge 0}{\iff}
        x = \sqrt[n \cdot m]{a^m} = \sqrt[n]{a}
    \end{align*}
\end{proof}

\begin{theorem}
    \begin{align*}
        \forall n \in \mathbb{N}: n \notdivisible 2 \;\;\; \forall m \in \mathbb{Z}: m \notdivisible 2 \;\;\; \forall a \in \mathbb{R} \backslash \{0\} \;\;\; 
        \sqrt[n]{a^m} = (\sqrt[n]{a})^m
    \end{align*}
\end{theorem}

\begin{proof}
    \hfill
    \begin{align*}
        &\text{Пусть } x = (\sqrt[n]{a})^m \text{, тогда} \\
        &x^n = ((\sqrt[n]{a})^m)^n \iff
        x^n = ((\sqrt[n]{a})^n)^m = a^m \iff
        x = \sqrt[n]{a^m} = (\sqrt[n]{a})^m
    \end{align*}
\end{proof}

\begin{theorem}
    \begin{align*}
        \forall n \in \mathbb{N}: n \divisible 2 \;\;\; \forall m \in \mathbb{Z}: m \divisible 2 \;\;\; \forall a > 0 \;\;\; 
        \sqrt[n]{a^m} = (\sqrt[n]{a})^m
    \end{align*}
\end{theorem}

\begin{proof}
    \hfill
    \begin{align*}
        &\text{Пусть } x = (\sqrt[n]{a})^m \text{, тогда} \\
        &x^n = ((\sqrt[n]{a})^m)^n \iff
        x^n = ((\sqrt[n]{a})^n)^m = a^m \overset{m \divisible 2}{\underset{a \neq 0}{\implies}}
        \left\{\begin{array}{l}
            x^n = a^m \\
            x > 0 \\
        \end{array}\right. \iff
        x = \sqrt[n]{a^m} = (\sqrt[n]{a})^m
    \end{align*}
\end{proof}

\begin{theorem}
    \begin{align*}
        \forall n \in \mathbb{N}: n \notdivisible 2 \;\;\; \forall a,b \in \mathbb{R}  \;\;\; a \le b \iff \sqrt[n]{a} \le \sqrt[n]{b}
    \end{align*}
\end{theorem}

\begin{proof}
    \hfill \\

    Пусть $x = \sqrt[n]{a} \land y = \sqrt[n]{b}$, тогда
    \begin{enumerate}
        \item $x = 0 \lor y = 0$ - очевидно
        \item $sign(x) \neq sign(y)$ - очевидно
        \item $x > 0 \land y > 0$:
        \begin{align*}
            &a-b = x^n - y^n = (x-y)\overbrace{(x^{n-1} + x^{n-2}y + x^{n-3}y^2 + \dots + y^{n-1})}^{> 0, \text{ так как } x \text{ и } y \text{ больше нуля}}
            \implies sign(a-b) = sign(x-y) \\
            &\implies sign(a-b) = sign(\sqrt[n]{a} - \sqrt[n]{b}) \\
        \end{align*}
        \item $x < 0 \land y < 0$:
        \begin{align*}
            &a-b = x^n - y^n = (x-y)\overbrace{(x^{n-1} + x^{n-2}y + x^{n-3}y^2 + \dots + y^{n-1})}^{> 0, \text{ *}}
            \implies sign(a-b) = sign(x-y) \\
            &\implies sign(a-b) = sign(\sqrt[n]{a} - \sqrt[n]{b}) \\
        \end{align*}
        \text{* в каждом слагаемом все множетили возводятся в степени одинаковой четности.} \\
        \text{Если степень четная, то такое слагаемое будет положительным, если нечетное то тогда} \\
        \text{слагаемое будет произведением двух отрицательных чисел и тоже будет положительным.}
    \end{enumerate}
\end{proof}

\begin{theorem}
    \begin{align*}
        \forall n \in \mathbb{N}: n \divisible 2 \;\;\; \forall a,b \ge 0 \;\;\; a \le b \iff \sqrt[n]{a} \le \sqrt[n]{b}
    \end{align*}
\end{theorem}

\begin{proof}
    \hfill \\

    Пусть $x = \sqrt[n]{a} \land y = \sqrt[n]{b}$, тогда
    \begin{enumerate}
        \item $x = 0 \lor y = 0$ - очевидно
        \item $x > 0 \land y > 0$:
        \begin{align*}
            &a-b = x^n - y^n = (x-y)\overbrace{(x^{n-1} + x^{n-2}y + x^{n-3}y^2 + \dots + y^{n-1})}^{> 0, \text{ так как } x \text{ и } y \text{ больше нуля}}
            \implies sign(a-b) = sign(x-y) \\
            &\implies sign(a-b) = sign(\sqrt[n]{a} - \sqrt[n]{b}) \\
        \end{align*}
    \end{enumerate}
\end{proof}
