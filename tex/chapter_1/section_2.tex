\section{Арифметический корень натуральной степени}

\begin{definition}[Четная степень]
    Арифметическим корнем натуральной четной степени $n$ из числа $a$ называется число $b$, если оно при возведении в степень $n$ равно числу $a$

    \[
        b =  \sqrt[n]{a} \overset{ n \divisible 2}{\underset{n \in \mathbb{N}}{\iff}} 
        \left\{
        \begin{array}{l}
            b \ge 0 \\
            b^n = a
        \end{array}\right.
    \]
\end{definition}

\begin{definition}[Нечетная степень]
    Арифметическим корнем натуральной нечетной степени $n$ из числа $a$ называется число $b$, если оно при возведении в степень $n$ равно числу $a$

    \[
        b =  \sqrt[n]{a} \overset{ n \notdivisible 2}{\underset{n \in \mathbb{N}}{\iff}} b^n = a
    \]
\end{definition}

\begin{example}
    $\sqrt[5]{625} = 4$; $\sqrt[3]{8} = 2$; $\sqrt[4]{-4} \in \varnothing$;
\end{example}

\label{thm:1.2.2}
\begin{theorem}[Теорема о корректности определения]
    \[
        \forall n \in \mathbb{N}: n \divisible 2 \;\;\; \forall a \ge 0 \;\;\; \exists! \; b \ge 0: b^n = a
    \]
\end{theorem}

\label{thm:1.2.3}
\begin{theorem}[Теорема о корректности определения]
    \[
        \forall n \in \mathbb{N}: n \notdivisible 2 \;\;\; \forall a \in \mathbb{R} \;\;\; \exists! \; b \in \mathbb{R}: b^n = a
    \]
\end{theorem}