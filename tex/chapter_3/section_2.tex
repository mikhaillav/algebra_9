\section{Иррациональные неравенства}

\begin{align*}
    \begin{array}{c}
        a, b \in A \\
        f \uparrow A
    \end{array}
    \hspace{1.5cm}
    a \le b \iff f(a) \le f(b) \\ \\
    \begin{array}{c}
        a, b \in A \\
        f \downarrow A
    \end{array}
    \hspace{1.5cm}
    a \le b \iff f(a) \ge f(b) \\
\end{align*}

\begin{remark}
    Возведение обоих частей неравенства в нечетную положительную степень всегда является равносильным преобразованием.
\end{remark}

\begin{remark}
    Если обе части неравенства неотрицательны, то их можно возводить в четную натуральную степень, но возможно потребуются ограничения по ОДЗ.
\end{remark}

\begin{example}
    $\sqrt{x - 2} > 3 \iff x - 2 > 9 \iff x > 11$
\end{example}

\begin{example}
    \begin{align*}
        &\sqrt{x - 2} \le 3 \iff
        \left\{\begin{array}{l}
            x - 2 \ge 0 \\
            x - 2  \le 9
        \end{array}\right. \iff
        \left\{\begin{array}{l}
            x \ge 2 \\
            x \le 11
        \end{array}\right. \iff
        x \in [2; 11]
    \end{align*}
\end{example}

\begin{remark}
    Если обе части неравенства неположительные числа, то неравенство можно возвести в четную натуральную степень, 
    но знак неравенства поменяется на противположгный
\end{remark}

\begin{remark}
    Если обе части неравенства имеют разный знак, то их нельзя возводить в четную натуральную степень, но при этом о выполнении неравенства
    можно непосредственно судить.
\end{remark}

\begin{example}
    \begin{align*}
        &\sqrt{x - 2} \le -x \iff
        \left\{\begin{array}{l}
            x \le 0 \\
            x - 2 \le x^2 \\
            x + 2 \ge 0
        \end{array}\right. \iff
        \left\{\begin{array}{l}
            x \in [-2; 0] \\
            (x + 1)(x - 2) \ge 0
        \end{array}\right. \iff
        \left\{\begin{array}{l}
            x \in [-2; 0] \\
            x \in (-\infty; -1] \cup [2; +\infty) \ge 0
        \end{array}\right. \\
        &\text{Ответ: } [-2; -1]
    \end{align*}
\end{example}

\begin{example}
    \begin{align*}
        &\sqrt{x - 2} \ge -x \iff
        \left[
            \begin{array}{l}
                \left\{\begin{array}{l}
                    x > 0 \\
                    x \ge -2
                \end{array}\right. \\
                \left\{\begin{array}{l}
                    x \le 0 \\
                    x + 2 \ge x^2
                \end{array}\right.
            \end{array}
        \right. \iff
        \left[
            \begin{array}{l}
                x \ge -2 \\
                \left\{\begin{array}{l}
                    x \le 0 \\
                    (x + 1)(x - 2) \le 0
                \end{array}\right.
            \end{array}
        \right. \iff
        \left[
            \begin{array}{l}
                x \ge -2 \\
                \left\{\begin{array}{l}
                    x \le 0 \\
                    x \in [-1; 2]
                \end{array}\right.
            \end{array}
        \right. \\
        &\text{Ответ: } [-1; +\infty)
    \end{align*}
\end{example}
