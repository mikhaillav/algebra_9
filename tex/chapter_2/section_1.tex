\section{Обратная функция}

\begin{remind}
    $f: A \rightarrow B$ иньекция, если $\forall x_1, x_2 \in A \;\; x_1 \neq x_2 \iff f(x_1) \neq f(x_2)$
\end{remind}

\begin{definition}[Сюръекция]
    $f: A \rightarrow B$ сюръекция, если $\forall y \in B \; \exists x \in A \; : \; y = f(x)$
\end{definition}

\begin{definition}[Биекция]
    $f: A \rightarrow B$ биекция, если она и иньекция и сюръекция одновременно.
\end{definition}

\begin{remark}
    Биекцию так же называют \textbf{взаимно однозначным соответствием}.
\end{remark}

\begin{definition}[Обратная функция]
    Функция $g: B \rightarrow A$ называется обратной к $f: A \rightarrow B$
    если выполняется условие:
    \begin{align*}
        \left\{\begin{array}{l}
            \forall x \in A \;\;\; g(f(x)) = x \\
            \forall y \in B \;\;\; f(g(y)) = y
        \end{array}\right.
    \end{align*}
\end{definition}

\begin{remark}
    $f: A \rightarrow B$ обратима тогда и только тогда когда биекция.
\end{remark}

\begin{example}
    $y=x$ обратная сама к себе
\end{example}

\begin{example}
    $y=2x + 1 \iff x = \frac{y - 1}{2}$ Тогда обратная функция к функции $f(x) = 2x + 1$ будет функция $g(x) = \frac{y - 1}{2}$
\end{example}

\begin{example}
    $f(x) = x^2 - 2x + 3$, при $A = \left(-\infty; 1\right]$
    \begin{align*}
        y = x^2 - 2x + 3 \iff x = 1 \pm \sqrt{y - 2} \overset{x \in A}{\iff} x = 1 - \sqrt{y - 2}
    \end{align*}
    Тогда $g(x) = 1 - \sqrt{y - 2}$ обартная к $f$ на множестве $A$
\end{example}

\begin{definition}
    Функция, обратная к функции $f(x)$ обозначается как $f(x)^{-1}$ 
\end{definition}

\begin{theorem}[О графике обратной функции]
    Графики прямой и обартной функции симметричны относительно прямой $y = x$.
\end{theorem}

\begin{proof}
    \begin{align*}
        &\forall (x; y) \; \forall f: A \rightarrow B, \; g: B \rightarrow A \;\;\; (x; y) \in \Gamma_f \iff \\
        &\iff  \left\{\begin{array}{l}
            x \in D(f) \\
            y = f(x)
        \end{array}\right. \iff
        \left\{\begin{array}{l}
            y \in D(g) \\
            g(y) = x 
        \end{array}\right. \iff
        (x; y) \in \Gamma_g
    \end{align*}

    Если $y = f(x)$ то $y \in D(g)$, ведь функция $f(x)$ как бы <<возвращает>> множество $B$ которое для функции $g$ является областью определения. \\
    А зная, что $g(x) = f(x)^{-1}$ мы можем применить функцию к обоим частям уравнения и получим в правой части <<чистый аргумент>> без функции.
\end{proof}

\break

\begin{theorem}[О монотонности обратной функции]
    \begin{align*}
        &\text{Если } f: A \rightarrow B, \; g: B \rightarrow A \land g(x) = f(x)^{-1} \text{ то} \\
        &\left\{\begin{array}{l}
            f \uparrow A \Rightarrow g \uparrow B \\
            f \downarrow A \Rightarrow g \downarrow B
        \end{array}\right.
    \end{align*}
\end{theorem}

\begin{proof}
    \begin{align*}
        1) & \; \forall x_1, x_2 \in A \;\; x_1 > x_2 \Rightarrow f(x_1) > f(x_2)  \\
        &\text{Предположим, } \exists y_1, y_2 \in B \;\; y_1 > y_2 \land g(x_1) \le g(x_2) \\
        &\left\{\begin{array}{l}
            g(y_1) = x_1 \\
            g(y_2) = x_2
        \end{array}\right. \Rightarrow
        x_1 \le x_2 \Rightarrow f(x_1) \le f(x_2) \; ?! \\ \\
        2) & \; \forall x_1, x_2 \in A \;\; x_1 > x_2 \Rightarrow f(x_1) < f(x_2)  \\
        &\text{Предположим, } \exists y_1, y_2 \in B \;\; y_1 > y_2 \land g(x_1) \ge g(x_2) \\
        &\left\{\begin{array}{l}
            g(y_1) = x_1 \\
            g(y_2) = x_2
        \end{array}\right. \Rightarrow
        x_1 \ge x_2 \Rightarrow f(x_1) \ge f(x_2) \; ?!
    \end{align*}
\end{proof}

