\section{Иррациональные уравнения}

\begin{align*}
    a = b &\iff f(a) = f(b) \\
    &\Rightarrow - \text{ верно всегда}\\
    &\Leftarrow - \text{ верно, если функция инъективна}
\end{align*}

\begin{definition}[Инъективность]
    Функция называется инъективной на множестве $A$ (инъекция на множестве $A$), если различным значениям аргумента из множества $A$, 
    функция сопостовляет различные значения функции.

    \begin{align*}
        &A \subset D(f) \\
        &\forall x_1, x_2 \in A \;\;\; x_1 \neq x_2 \Rightarrow f(x_1) \neq f(x_2)
    \end{align*}
\end{definition}

\begin{remark}
    На основании теоремы "док-во от обратного":
    \begin{align*}
        &(x_1 \neq x_2 \Rightarrow f(x_1) \neq f(x_2)) \iff (f(x_1) = f(x_2) \Rightarrow x_1 = x_2)
    \end{align*}
\end{remark}

\hfill \newline

\begin{remark}
    График инъективной функции пересекается произвольной горизонтальной прямой не больше одного раза.
\end{remark}

\begin{theorem}[Достаточное условие инъективности]
    Если функция строго монотонна на множестве $A$, то она инъективна на множестве $A$.
\end{theorem}

\begin{proof}
    \begin{align*}
        &\text{Функция строго монотонна, значит } \\
        &\left[\begin{array}{l}
            x < y \iff f(x) < f(y) \\
            x > y \iff f(x) > f(y) 
        \end{array}\right. \Rightarrow
        (x \neq y \iff f(x) \neq f(y))
    \end{align*}
\end{proof}

\begin{remark}
    Обратное неверно!
\end{remark}

\begin{remark}
    Степенная функция с нечетным натуральным показателем -- инъекция на всей вещестенной оси, так как всегда возрастает. 
\end{remark}

\begin{remark}
    Возведение обоих частей уравнения в нечетную натуральную степень всегда будет равносильным преобразованием.
\end{remark}

\begin{remark}
    Степенная функция с четным натуральным показателем не является инъекцией на всей вещестенной оси. 
\end{remark}

\begin{example}
    \begin{align*}
        \text{Если } f(x) = x^4; \; x_1 = 2; \; x_2 = -2, \text{ то } x_1 \neq x_2, \text{ но } f(x_1) = f(x_2) = 16
    \end{align*}
\end{example}

\begin{remark}
    Степенная функция с четным натуральным показателем является инъекцией отдельно на множестве $(-\infty; \; 0]$ и $[0; \; +\infty )$ 
\end{remark}

\begin{remark}
    При возведении обоих частей уравнения в четную натуральную степень могут появится лишние решения.
\end{remark}

\begin{example}
    \begin{align*}
        &\sqrt{x + 2} = x \iff x + 2 = x^2 \iff x^2 - x - 2 = 0 \iff \\
        &\left[\begin{array}{l}
            \uwave{x = -1} \leftarrow \text{лишний корень}\\
            x = 2
        \end{array}\right.
    \end{align*}
\end{example}

\begin{remark}
    Если можем гарантировать, что обе части уравнения имеют одинаковый знак, то возведение их в четную натуральную степень
    будет равносильным преобразованием.
\end{remark}

\begin{example}
    \begin{align*}
        &\sqrt{x + 2} = x \underset{x \ge 0}{\iff} x + 2 = x^2 \iff x^2 - x - 2 = 0 \iff \\
        &\left[\begin{array}{l}
            x = -1 \\
            x = 2
        \end{array}\right. \underset{x \ge 0}{\iff}
        x = 2
    \end{align*}
\end{example}

\begin{example}
    \begin{align*}
        &\sqrt{2x + 7} - \sqrt{x + 3} = 1 \iff \sqrt{2x + 7} = \sqrt{x + 3} + 1 \iff \\
        &2x + 7 = 1 + 2\sqrt{x + 3} + x + 3 \iff 2\sqrt{x + 3} = x + 3 \iff\\
        &4x + 12 = x^2 + 6x + 9 \iff x^2 - 2x - 3 = 0 \iff
        \left[\begin{array}{l}
            x = 1 \\
            x = -3
        \end{array}\right.
    \end{align*}
\end{example}